\documentclass[a4paper,11pt]{article}

%%%%%%%%%%%%%%%%%%%%%%%%%%%%%%%%%%%%%%%%%%%%%%%%%%%%%%%%%%%%%%%%%%%%%%%%
% Paquetes utilizados
%%%%%%%%%%%%%%%%%%%%%%%%%%%%%%%%%%%%%%%%%%%%%%%%%%%%%%%%%%%%%%%%%%%%%%%%
% Soporte para el lenguaje español
\usepackage{textcomp}
\usepackage[utf8]{inputenc}
\usepackage[T1]{fontenc}
\usepackage[spanish]{babel}

% Graficos
\usepackage{graphicx}
\usepackage{rotating}
\usepackage{pdflscape}

% Encabezados y pies de pagina
\usepackage{fancyhdr}
\setlength{\headheight}{15.2pt}
\pagestyle{fancy}
\fancyhf{}
\lhead{(75.99) Trabajo Profesional}
\rhead{Plan de Proyecto}
\cfoot{\thepage}

\begin{document}

%%%%%%%%%%%%%%%%%%%%%%%%%%%%%%%%%%%%%%%%%%%%%%%%%%%%%%%%%%%%%%%%%%%%%%%%
% Titulo
%%%%%%%%%%%%%%%%%%%%%%%%%%%%%%%%%%%%%%%%%%%%%%%%%%%%%%%%%%%%%%%%%%%%%%%%

\thispagestyle{empty}

\begin{titlepage}

\newcommand{\HRule}{\rule{\linewidth}{0.5mm}}
\newenvironment{bottompar}{\par\vspace*{\fill}}{\clearpage}

\center

\textsc{\LARGE Universidad de Buenos Aires}\\[0.5cm]
\textsc{\Large Facultad de Ingeniería}\\[1.5cm]

\includegraphics[scale=0.5]{src/common/logo.png}\\[1cm]


\textsc{\large (75.99) Trabajo Profesional}\\[0.25cm]
\HRule \\[0.4cm]
{\huge \bfseries Plan de Proyecto}
\HRule \\[0.4cm]

{\large quecurso.com.ar}

\begin{bottompar}
\flushleft
{\bfseries Integrantes:}

Arana, Andrés          - P. 86.203

Piano, Sergio          - P. 85.191
\end{bottompar}

\end{titlepage}

%%%%%%%%%%%%%%%%%%%%%%%%%%%%%%%%%%%%%%%%%%%%%%%%%%%%%%%%%%%%%%%%%%%%%%%%
% Documento
%%%%%%%%%%%%%%%%%%%%%%%%%%%%%%%%%%%%%%%%%%%%%%%%%%%%%%%%%%%%%%%%%%%%%%%%

\section{Introducción}
\baselineskip=18pt

\section{Propósito o justificación del proyecto}

  Hay aproximadamente 9000 alumnos de grado cursando en la Facultad de
  Ingeniería de la Universidad de Buenos Aires. Todos ellos se enfrentan al
  mismo problema: Organizar la cursada es difícil, demanda una planificación y
  gestión que es lateral a las tareas académicas. Desde la selección de
  materias a cursar, que implica el análisis de feedback por parte de otros
  alumnos en foros públicos, la utilización de grillas de cálculo para la
  revisión de correlatividades, y un proceso posterior manual de
  calendarización; hasta el armado y coordinación de grupos de trabajos
  prácticos o el control de fechas de entregas y evaluaciones.

  El problema existe, y la evidencia de esto es que hay varias herramientas
  aisladas e informales creadas por alumnos para ayudar a la gestión de la
  cursada, como ser la InfoGrilla (un excel con macros que resuelve el problema
  de correlatividades para la carrera de Ingeniería en Informática) o proyectos
  como PlaniFI.

\section{Objetivo}

  Desarrollar una herramienta de organización y autogestión de la cursada para
  el alumnado de la facultad de Ingeniería de la Universidad de Buenos Aires.
  El proyecto tiene una duración de 8 meses que comienza en Noviembre 2015 y
  finalizará en Junio 2016. 

\section{Interesados}

  Los princpales interesados son:

\begin{description}

  \item[Alumnado de la facultad] 

    Considerado beneficiario del sistema, afecatado positivamente. Su rol
    dentro de la facultad es de usuario, es de tipo interno debido a quién será
    el usuario principal. La necesidad de información que maneja es amplia y la
    manera de contactar es a través de su dispositivo o de su email.

  \item[La facultad]

    Considerado beneficiario del sistema, lo afecta positivamente. Tiene baja
    necesidad de información, y puede llegar a tener un impacto alto en función
    de la disponibilidad de acceso a servicios. Los contactos que tenemos son
    por mail. 
  
  \item[Los profesores]
    
    Pueden llegar a tener un impacto negativo. Bajo nivel de necesidad de
    información, la manera que tenemos de contactarnos es a través del mail.

  \item[Nosotros]
    
    Considerados beneficiarios del sistema, siendo ademas de los creadores del
    sistema también alumnado de la facultada de ingeniería.

\end{description}

\section{Requerimientos}

  Para la construcción del proyecto se deberá diseñar y desarrollar una
  aplicación web que permita gestionar la cursada del alumno en varios
  aspectos. Los mismos le presentarán herramientas para la selección de
  materias a cursar, el armado y gestión de los grupos de estudio y trabajos
  prácticos, un sistema de alertas en fechas importantes y un sistema de puntos
  para motivar al usuario a ir cumpliendo metas y objetivos.

\section{Supuestos}

  Los alumnos que utilizarán el sistema contarán con un dispositivo móvil.
  Acceso a servicios que permitan conocer la información de las cursadas, como
  ser las materias y sus horarios.  El alumno va a cargar su perfil informando
  la cantidad de materias que tiene realizadas y las notas de cada una de
  ellas.

\section{Limitaciones}

  Nos limitamos a realizar el sistema para cubrir las necesidades de los
  alumnos de la Facultad de Ingeniería de Buenos Aires.  Nos limitamos a
  quienes tengan dispositivos móviles que dispongan de un sistema operativo tal
  como android que soporte la instalación de navegadores tales como google
  chrome.

\section{Cronograma}

  El milestone va a tener una duración de 8 meses, con iteraciones separadas
  cada 15 días, de ésta manera el milestone contará de 16 iteraciones.
  Considerando que cada uno de los desarrolladores le dedicarán exclusivamente
  12 horas semanales. Siendo 24 hpras a la quincena. Son 16 quincenas que cada
  uno cubrirá con 384 horas totales. Disponiendo de dos desarrolladores se
  cubren las 720 horas para dar por finalizado el desarrollo del sistema.

\section{Costos}

  Vamos a estar destinando 720 horas divididas en dos desarrolladores seniors.
  Cada uno de ellos trabajará 360hs dedicada a cada una de sus asignaciones.
  El costo de mantenimiento de los servidores se acentuará en el caso que se
  extienda el periodo de licencia gratuita que ofrecen los servicios de
  hosting.  El costo estimativo de los servidores sería de 20 dolares al año
  que se completaría con los sistemas de sponsors de publicidad online.


\section{Riesgos}

\newpage
\addtolength{\topmargin}{-1.575in}
\thispagestyle{empty}
\begin{table}[h!t]
\centering
\rotatebox{90}{
  \begin{tabular}{ | l | p{3cm} | p{1cm} | p{2cm} | p{2cm} | p{1cm} | p{1cm} | p{1cm} | p{3cm} | p{2cm} | p{2cm} | p{1cm} | p{1cm} | p{1cm} | }
  \hline
  \# & Riesgo & Tipo & Causas & Consecuencias & Probabildad & Impacto & Exposición & Plan de Respuesta & Umbral & Plan de Contingencia & Estado & Responsable & Observaciones\\ \hline
  1 &
  Dado que el alumno tiene que cargar su situción actual dentro de la facultad para dar inicio a los flujos dentro del sistema, no tendríamos registro de tal alumno. &
  Amenaza &
  Tener que llenar un perfil del alumno dentro de la facultad para poder tener un feedback de su situación actual &
  No tener al alumno registrado dentro del sistema &
  Alta & Alta & Alta &
  Mitigar, vamos a mostrar una buena experiencia de usuario para poder atrapar al alumno &
  Si no se consigue un minimo de 10 usuarios nuevos por mes. &
  pensamos que otras alternativas de marketing pueden llegar a aplicar. &
  Sin suceder &
  Lider de proyecto & \\ 
  2 &
  Dado que no haya un alumno que indique al sistema las fechas importantes, tales como examenes de parciales, entregas de trabajos prácticos o fechas de finales no se generarían alertas para las fechas mencionadas &
  Amenaza &
  Falta de notificación de parte de alguno de los alumnos de las materias en cuanto a las fechas importantes &
  No se generarían alertas para dichas fechas.  &
  Alta & Alta & Alta &
  Aceptar. Considerando que los alumnos no van a informar la fecha, se desactivará la alerta en éste caso para la materia en la que no se tiene información.  &
  - & &
  Sin suceder &
  Lider de proyecto & \\ 
  \hline
\end{tabular}
}
\end{table}

\end{document}
