\documentclass[a4paper,11pt]{article}

%%%%%%%%%%%%%%%%%%%%%%%%%%%%%%%%%%%%%%%%%%%%%%%%%%%%%%%%%%%%%%%%%%%%%%%%
% Paquetes utilizados
%%%%%%%%%%%%%%%%%%%%%%%%%%%%%%%%%%%%%%%%%%%%%%%%%%%%%%%%%%%%%%%%%%%%%%%%
% Soporte para el lenguaje español
\usepackage{textcomp}
\usepackage[utf8]{inputenc}
\usepackage[T1]{fontenc}
\usepackage[spanish]{babel}

% Gráficos
\usepackage{graphicx}

% Encabezados y pies de pagina
\usepackage{fancyhdr}
\setlength{\headheight}{15.2pt}
\pagestyle{fancy}
\fancyhf{}
\lhead{(75.99) Trabajo Profesional}
\rhead{Manual de Proyecto}
\cfoot{\thepage}

% Courier para términos técnicos
\usepackage{courier}

\begin{document}

%%%%%%%%%%%%%%%%%%%%%%%%%%%%%%%%%%%%%%%%%%%%%%%%%%%%%%%%%%%%%%%%%%%%%%%%
% Titulo
%%%%%%%%%%%%%%%%%%%%%%%%%%%%%%%%%%%%%%%%%%%%%%%%%%%%%%%%%%%%%%%%%%%%%%%%

\thispagestyle{empty}

\begin{titlepage}

  \newcommand{\HRule}{\rule{\linewidth}{0.5mm}}
  \newenvironment{bottompar}{\par\vspace*{\fill}}{\clearpage}

  \center

  \textsc{\LARGE Universidad de Buenos Aires}\\[0.5cm]
  \textsc{\Large Facultad de Ingeniería}\\[1.5cm]

  \includegraphics[scale=0.5]{src/common/logo.png}\\[1cm]


  \textsc{\large (75.99) Trabajo Profesional}\\[0.25cm]
  \HRule \\[0.4cm]
  {\huge \bfseries Manual de Proyecto}
  \HRule \\[0.4cm]

  {\large quecurso.com.ar}

  \begin{bottompar}
    \flushleft
    {\bfseries Integrantes:}

    Arana, Andrés          - P. 86.203

    Piano, Sergio          - P. 85.191
  \end{bottompar}

\end{titlepage}

%%%%%%%%%%%%%%%%%%%%%%%%%%%%%%%%%%%%%%%%%%%%%%%%%%%%%%%%%%%%%%%%%%%%%%%%
% Documento
%%%%%%%%%%%%%%%%%%%%%%%%%%%%%%%%%%%%%%%%%%%%%%%%%%%%%%%%%%%%%%%%%%%%%%%%

\section{Introducción}
\baselineskip=18pt

El presente documento sumariza las metodologías, los procesos, las herramientas
y las técnicas utilizadas durante el desarrollo del proyecto.

\section{Metodología}

La metodología utilizada está basada en SCRUM, y adaptada a las
particularidades de un equipo de trabajo pequeño que no está constantemente
trabajando en el mismo horario ni lugar.

Algunas diferencias y particularidades con respecto a SCRUM son las siguientes:

\begin{description}

  \item[Product Owner]

    El rol del product owner es llevado a cabo por ambos integrantes del
    equipo. Debido a que el sistema a desarrollar es de consumo masivo y con
    releases intermedios parciales de tipo MVP, los intereses del usuario son
    determinados a través de analytics de uso, utilizando técnicas de A/B
    testing o similares para detectar y validar posibles requisitos.

  \item[Scrum Master]

    El rol de scrum master se rotará en cada sprint entre los integrantes del
    equipo.

  \item[Sprints]

    Se decidió realizar sprints de dos semanas, con una única reunión de cierre
    que incluye las tareas desarrolladas normalmente en las reuniones de
    planning y retrospective de la metodología al finalizar cada una de las
    sprints.

    En cada reunión de cierre se analizan y repriorizan las tareas pendientes
    de la sprint, se reprioriza el backlog, se seleccionan en función de la
    velocidad del equipo las tareas de la siguiente sprint y se realiza la
    revisión retrospectiva sobre la sprint terminada.

  \item[Daily scrums]

    Se decidió evitar la realización de las daily scrums, debido a que el
    equipo no va a estar colocado espacial ni temporalmente excepto para las
    reuniones de planning y retrospective.

\end{description}

\section{Trazabilidad}

En esta sección se detallan las herramientas y procesos relacionados a dónde y
cómo se registran distintos aspectos de la gestión del proyecto.

\subsection{Trazabilidad de Eventos Metodológicos}

De acuerdo a lo descripto en la sección de Metodología existen ciertos eventos
metodológicos cuya trazabilidad debe ser asegurada:

\begin{description}

  \item[Planning meetings]

    Las minutas de las reuniones bisemanales de planificación de sprints se
    registran en este mismo repositorio, dentro de la carpeta
    \texttt{memo/planning}. Cada minuta es un archivo diferente dentro de esa
    carpeta, cuyo nombre debe tener el formato \texttt{YYYYMMDD.md}. Las
    minutas se confeccionan a partir de un template en la misma carpeta con el
    nombre \texttt{\_template\_.md}.

    La gestión de las tareas se registra en los boards de trello como se
    describe en la sección de Trazabilidad de Requerimientos.

  \item[Retrospective meetings]

    Las minutas de las reuniones retrospectivas de sprints también se registran
    en este mismo repositorio, dentro de la carpeta \texttt{memo/retro}. Cada
    minuta es un archivo diferente dentro de esa carpeta, cuyo nombre debe
    tener el formato \texttt{YYYYMMDD.md}. Las minutas se confeccionan a partir
    de un template en la misma carpeta con el nombre \texttt{\_template\_.md}.

\end{description}

\subsection{Trazabilidad de Requisitos, Bugs y Technical Spikes}

Los requisitos, bugs y spikes se registran en un backlog en un board de trello
(https://trello.com/b/nLy9mjZL/backlog). Dentro de este board se crean cards
categorizadas en 5 listas priorizadas. Cada lista contiene una tarjeta
denominada \texttt{TEMPLATE} con la que confeccionar las cards de cada tipo.
Las listas son las siguientes:

\begin{description}

  \item[Admin]

    Esta lista contiene las tareas administrativas, como gestión de documentos,
    armado de presentaciones y similares. Se decidió incluirlas en la
    planificación de la sprint y en el ciclo de vida del proyecto debido a que
    el equipo es reducido y este tipo de tareas influye de manera marcada en la
    velocidad de la sprint.

  \item[Epics]

    Esta lista contiene las epic stories del proyecto, funcionalidades de alto
    nivel que luego se analizan en más detalle generando las user stories
    asociadas.

  \item[Stories]

    Esta lista contiene las user stories del proyecto. Cada user story es una
    funcionalidad completa e indivisible que puede entrar en un release de
    manera atómica, está asociada con una epic story padre y con una o más
    cards relacionadas.

  \item[Bugs]

    Esta lista contiene los bugs detectados en el proyecto. Cada bug tiene en
    el título de la card la criticidad de acuerdo al siguiente criterio:

    \begin{description}

      \item[Minor]

        Errores tipográficos o cosméticos que no interrumpen un flujo de
        usuario.

      \item[Major]

        Errores funcionales que interrumpen la operación de la aplicación y que
        tienen workaround dentro de un flujo de usuario.

      \item[Blocker]

        Errores funcionales que interrumpen un flujo de usuario y que no tienen
        workaround.

    \end{description}

  \item[Spikes]

    Esta lista contiene las tareas de investigación, diseño o prototipado.

\end{description}

Durante cada reunión de planning se seleccionan de las listas de Admin,
Stories, Bugs y Spikes las cards que se desarrollaran en la sprint en función
de la velocidad promedio de todas las sprints. Todas las tareas seleccionadas
deben tener una estimación de complejidad en story points dentro de la escala
de Fibonacci. Las tareas seleccionadas se mueven al board de trello en
https://trello.com/b/fWOELyE0/sprint, y se les asigna un label en función del
tipo original al que pertenecían (Admin, Stories, Bugs, Spikes).

Durante la sprint, se desarrollan las tareas del board de sprint. Este board
tiene columnas a través de las cuales se mueven las cards de la sprint para
representar el avance de cada tarea. Las columnas son las siguientes:

\begin{description}

  \item[Backlog]

    Esta lista contiene las cards de la sprint que no están en progreso ni
    terminadas.

  \item[In Progress]

    Esta lista contiene las tareas que están en progreso. Obligatoriamente
    deben tener al menos un integrante asignado.

  \item[Review]

    Esta lista contiene las tareas que están en revisión. En las cards de tipo
    story o bug este estado implica que existe un Pull Request (ver sección de
    Versionado) asociado a la tarea, y que la misma está lista para incorporar
    al entorno productivo. En otro tipo de tareas implica que está listo para
    mostrar y discutir el entregable asociado.

  \item[Done]

    Esta lista contiene las tareas que están listas. En el caso de las cards de
    tipo story o bug, esto implica que la misma fue incorporada al entorno
    productivo. En otro tipo de tareas implica que el entregable asociado está
    revisado y aceptado por las partes interesadas.

\end{description}

En cada reunión de cierre de sprint se registra en un google spreadsheet en la
carpeta de google drive del proyecto denominado metrics las mediciones de la
sprint, como son los story points totales implementados en la sprint y la
cantidad de horas consumidas. Es este documento el que se utiliza para calcular
la velocidad promedio de las sprints.

\subsection{Trazabilidad de Costos}

Los costos del proyecto se registran en horas de trabajo utilizando Toggl
(https://www.toggl.com/app) a través del addon de togglbutton. Cuando se dedica
tiempo a cualquier card del backlog o de la sprint se utiliza el togglbutton de
la card para registrar el inicio y el fin del trabajo.

En la reunión de cierre de la sprint cada integrante debe utilizar el reporte
de horas de la herramienta para obtener la cantidad de horas totales que se
utilizó en cada story. Estas horas se registran en el mismo google spreadsheet
de metrics descripto en la sección anterior. Con este documento se calculan y
controlan las horas consumidas y restantes para realizar una gestión de costos
eficaz.

\section{Versionado}

El proyecto es versionado con Git en GitHub (https://github.com/) en 3
repositorios públicos dentro de la organización quecurso
(https://github.com/quecurso):

\begin{description}

  \item[https://github.com/quecurso/docs]

    Contiene los documentos relacionados al proyecto, como el brief de
    proyecto, el presente manual, las minutas de reuniones y similares.

  \item[https://github.com/quecurso/backend]

    Contiene la aplicación web que expone los servicios REST que constituyen el
    engine de estado persistente de la aplicación.

  \item[https://github.com/quecurso/frontend]

    Contiene el cliente estático HTML5 a través del cual los usuarios
    interactúan con la aplicación, ya sea desde plataformas desktop o mobile.

\end{description}

Tanto el repositorio de backend como el de frontend siguen la misma estrategia
de versionado:

\begin{description}

  \item[master]

    Branch en el que se encuentra el código que está incorporado al entorno
    productivo. Todos los commits de este branch son automáticamente
    incorporados al entorno productivo una vez que la regresión automatizada se
    completa exitosamente (ver la sección de Continuous Delivery).

  \item[f-xyz-title]

    Branches que contienen commits relacionados a una story particular. Por
    ejemplo, para una story con card id 23 y título "Información de perfil", el
    branch correspondiente se llamaría \texttt{f-23-profile-information}.

    Los commits de estos branches no se incorporan a ningún entorno, aunque si
    se ejecuta la regresión automatizada por cada commit publicado en el
    repositorio central.

    Estos branches de features se mantienen actualizados con respecto al master
    a través de la herramienta de \texttt{git rebase}.

    Una vez que el feature está completo, el branch se publica en el
    repositorio central y se crea un Pull Request contra \texttt{master} con el
    número de card y el título de la story (siguiendo el ejemplo anterior, el
    título del pull request sería "23 - Información de Perfil". El Pull Request
    debe contener un link a la card de trello de la user story y una
    descripción de los cambios realizados. Por otro lado, en trello se debe
    mover la tarjeta al estado Review, asignar a otro de los miembros a la
    tarjeta y agregar un comentario con la url en donde está el Pull Request.

  \item[b-xyz-title]

    Branches similares a \texttt{f-xyz-title} pero relacionados con bugs en vez
    de stories. Su manejo es idéntico a los branches de features.

\end{description}

Con respecto al repositorio de documentos, los commits se hacen sobre branches
similares a los branches de features pero cuyo nombre es \texttt{xyz-title},
donde xyz es el número de la card administrativa que originó el cambio de
documentación y title es el título de la misma. El mecanismo de gestión de
integración de estos branches es el mismo que en los otros repositorios
(rebases continuos, Pull Requests, links en las cards, etc.).

\section{Continuous Delivery}

Se utilizará Travis (https://travis-ci.org/) para integrar continuamente el
código, ejecutar las pruebas unitarias y de integración automáticas de ambos
proyectos y realizar los despliegues a todos los entornos en el caso de que el
commit sea en el branch \texttt{master}.

La aplicación se ejecutará sobre la plataforma de Heroku
(https://www.heroku.com/), y Travis se encargará de realizar los despliegues a
los distintos entornos, uno productivo y otro de staging.

\end{document}
