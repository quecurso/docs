\documentclass[a4paper,11pt]{article}

%%%%%%%%%%%%%%%%%%%%%%%%%%%%%%%%%%%%%%%%%%%%%%%%%%%%%%%%%%%%%%%%%%%%%%%%
% Paquetes utilizados
%%%%%%%%%%%%%%%%%%%%%%%%%%%%%%%%%%%%%%%%%%%%%%%%%%%%%%%%%%%%%%%%%%%%%%%%
% Soporte para el lenguaje español
\usepackage{textcomp}
\usepackage[utf8]{inputenc}
\usepackage[T1]{fontenc}
\usepackage[spanish]{babel}

% Graficos
\usepackage{graphicx}

% Encabezados y pies de pagina
\usepackage{fancyhdr}
\setlength{\headheight}{15.2pt}
\pagestyle{fancy}
\fancyhf{}
\lhead{(75.99) Trabajo Profesional}
\rhead{Brief de Proyecto}
\cfoot{\thepage}

\begin{document}

%%%%%%%%%%%%%%%%%%%%%%%%%%%%%%%%%%%%%%%%%%%%%%%%%%%%%%%%%%%%%%%%%%%%%%%%
% Titulo
%%%%%%%%%%%%%%%%%%%%%%%%%%%%%%%%%%%%%%%%%%%%%%%%%%%%%%%%%%%%%%%%%%%%%%%%

\thispagestyle{empty}

\begin{titlepage}

\newcommand{\HRule}{\rule{\linewidth}{0.5mm}}
\newenvironment{bottompar}{\par\vspace*{\fill}}{\clearpage}

\center

\textsc{\LARGE Universidad de Buenos Aires}\\[0.5cm]
\textsc{\Large Facultad de Ingeniería}\\[1.5cm]

\includegraphics[scale=0.5]{logo.png}\\[1cm]


\textsc{\large (75.99) Trabajo Profesional}\\[0.25cm]
\HRule \\[0.4cm]
{\huge \bfseries quecurso.com.ar}\\[0.4cm]
\HRule \\[0.5cm]

{\large \today}

\begin{bottompar}
\flushleft
Arana, Andrés          - P. 86.203

Piano, Sergio          - P. 85.191
\end{bottompar}

\end{titlepage}

%%%%%%%%%%%%%%%%%%%%%%%%%%%%%%%%%%%%%%%%%%%%%%%%%%%%%%%%%%%%%%%%%%%%%%%%
% Documento
%%%%%%%%%%%%%%%%%%%%%%%%%%%%%%%%%%%%%%%%%%%%%%%%%%%%%%%%%%%%%%%%%%%%%%%%

\section*{Introducción}

\baselineskip=18pt
Hay aproximadamente 9000 alumnos de grado cursando en la Facultad de Ingeniería
de la Universidad de Buenos Aires. Todos ellos se enfrentan al mismo problema:
Organizar la cursada es difícil, demanda una planificación y gestión que es
lateral a las tareas académicas. Desde la selección de materias a cursar, que
implica el análisis de feedback por parte de otros alumnos en foros públicos,
la utilización de grillas de cálculo para la revisión de correlatividades, y un
proceso posterior manual de calendarización; hasta el armado y coordinación de
grupos de trabajos prácticos o el control de fechas de entregas y evaluaciones.

El problema existe, y la evidencia de esto es que hay varias herramientas
aisladas e informales creadas por alumnos para ayudar a la gestión de la
cursada, como ser la InfoGrilla (un excel con macros que resuelve el problema
de correlatividades para la carrera de Ingeniería en Informática) o proyectos
como PlaniFI.

\section*{{\Large{}El proyecto}}

Proponemos armar una aplicación web junto con un componente mobile, que le
permita a los alumnos gestionar su cursada en varios aspectos, como la
selección de materias a cursar, el armado y operación de los grupos de estudio
y de trabajos prácticos y alertas en fechas significativas, enmarcado en un
sistema con funcionalidades de gamification.

\subsubsection*{\textbf{Módulos y funcionalidades principales}}

Los módulos principales del sistema son los siguientes:

\begin{itemize}
    \item Módulo de datos personales y de perfil, incluyendo carrera y
      orientación.

    \item Módulo de gestión de contactos, en donde un alumno puede indicar que
      otro es un compañero bajo aceptación de este otro, con integración con
      redes sociales para network discovery.

    \item Módulo de registro de materias aprobadas, materias que el alumno
      tiene interés por cursar y materias actualmente cursando.

    \item Módulo de calificación de materias, con ratings en diferentes
      aspectos importantes como calidad de clases teóricas, prácticas y
      dificultad.

    \item Módulo de calendarización de cursada, con un algoritmo de generación
      automática en base a correlatividades, materias, intereses y
      calificaciones, con posibilidad de generación manual.

    \item Módulo autogestionado de información de fechas de evaluación y
      entregas de trabajos prácticos. Los alumnos pueden registrar fechas
      importantes, y es la propia comunidad de alumnos a través de un sistema
      de reputación y validación cruzada la que controla estas registraciones
      generales.

    \item Módulos de recordatorios personales, para registrar fechas de firma
      de libretas y otros trámites.

    \item Módulo de notificaciones, con push notifications, emails y otros
      canales de alertas.

    \item Módulo de gamification, logros, puntos de experiencia y niveles para
      fomentar la participación del alumnado, tanto en el uso del sistema como
      en el control de información cargada por otros usuarios.

    \item Módulo de integración con aplicaciones de calendarios, como google
      calendar.

    \item Módulo de recomendación de materias a través de sistema inteligente
      de clasificación en base al perfil académico del alumno.

\end{itemize}

\section*{{\Large{}Quiénes somos}}

Los integrantes del equipo de desarrollo del proyecto son:

\begin{itemize}
    \item Andrés Arana, estudiante de Ingeniería en Informática, padrón 86.203.
      Trabaja en el área de sistemas hace 11 años, desempeñando tareas de
      desarrollo, análisis de requerimientos y planificación de proyectos. En
      este momento trabaja en un emprendimiento propio de producto de software.

    \item Sergio Piano, estudiante de Ingeniería en Informática, padrón 85.191.
      Trabaja en sistemas hace 9 años, desempeñando tareas de liderazgo técnico
      y desarrollo. Trabajando actualmente en mercadolibre.
\end{itemize}

\section*{{\Large{}Proyectos relacionados}}

En el 1° cuatrimestre de 2015 se desarrolló para la materia 75.47 - Taller de
Desarrollo de Proyectos II un trabajo práctico denominado ``Campus Virtual''
cuya funcionalidad está relacionada al proyecto que estamos presentando. Es de
nuestro interés integrar lo expuesto anteriormente con lo ya desarrollado en
este proyecto, orientado fundamentalmente a la gestión de foros y grupos de
comunicación.

\end{document}
